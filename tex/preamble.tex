%!TeX encoding=utf8

\newcommand{\authA}{Daniel Scheiermann}
\newcommand{\matA}{}
\newcommand{\grpnr}{}
\newcommand{\Versuchsnummer}{}
\newcommand{\Versuchsname}{}

%%%%%%%%%%%%%%%%%%Capitalized Color - start%%%%%%%%%%%%%%%%%%%%%%%%
\usepackage{xparse}
\usepackage{xcolor}
\definecolor{luh_blue}{RGB}{0,80,155} %the blue of the luh logo


\ExplSyntaxOn
\NewDocumentCommand{\colorcap}{ O{blue} m }
 {
  \sheljohn_colorcap:nn { #1 } { #2 }
 }

\tl_new:N \l__sheljohn_colorcap_input_tl
\cs_new_protected:Npn \sheljohn_colorcap:nn #1 #2
 {
  % store the string in a variable for usage with \regex_replace_all:nnN
  \tl_set:Nn \l__sheljohn_colorcap_input_tl { #2 }
  \regex_replace_all:nnN
   { ([A-Z]+) } % search a capital letter (or more)
   { \c{textcolor}\cB\{#1\cE\}\cB\{\1\cE\} } % replace the match with \textcolor{#1}{<match>}
   \l__sheljohn_colorcap_input_tl
  \tl_use:N \l__sheljohn_colorcap_input_tl
 }
\ExplSyntaxOff
%%%%%%%%%%%%%%%%%%Capitalized Color - end%%%%%%%%%%%%%%%%%%%%%%%%
%---configure pagelayout
\KOMAoptions{ % read documentation for KOMAScript providing the documentclass scrartcl
DIV=11,
BCOR=0mm,
paper=a4,
fontsize=12pt,
parskip=half,
twoside=false,
titlepage=true
}

%\usepackage[ %Set linespacing
%singlespacing %onehalfspacing,doublespacing
%]{setspace}

\usepackage[headsepline,footsepline,automark,pagestyleset=KOMA-Script,markcase=ignoreuppercase]{scrlayer-scrpage} %Configure headline and footer

\clearscrheadings
\setlength{\headheight}{3.5\baselineskip}

% \ihead[]{\authA~(\matA)}
\ihead[]{\authA~\matA}
\ohead{\includegraphics[height=13pt]{IMAGE/luh_logo.png}
%\\ $ $ \grpnr
}
%\Versuchsnummer


%\clearpairofpagestyles
\cfoot{\vspace{-0.75cm}\pagemark}
\pagestyle{scrheadings}

%better positioning of floatings
\renewcommand{\floatpagefraction}{.75} % standard: .5
\renewcommand{\textfraction}{.1} % standard: .2
\renewcommand{\topfraction}{.8} % standard: .7
\renewcommand{\bottomfraction}{.5} % standard: .3
\setcounter{topnumber}{3} % standard: 2
\setcounter{bottomnumber}{2} % standard: 1
\setcounter{totalnumber}{5} % standard: 3

%---Language and umlauts
\usepackage[utf8]{inputenc} %Set UTF-8 encoding, enables ä,ö,ü etc.
% \DeclareUnicodeCharacter{2212}{-}
\usepackage[english]{babel} %Set document language to ngerman (new german)
%\usepackage[% improved hyphenation
%expansion=true,
%protrusion=true
%]{microtype}
\usepackage{subcaption}
%\usepackage{subfig} %subcaption includes subfig
%\captionsetup[subfigure]{list=true, font=large, labelfont=bf,
%labelformat=brace, position=top}

%---Mathmatics (AMS packages )
\usepackage{amsmath} %generell math enviorments e.g. align
\usepackage{amsfonts}
\usepackage{amssymb}
%\usepackage{amsthm} %math theorems
%\usepackage{upgreek}%provide special form of greek letters e.g. \upmu
\usepackage{float}
%---Units
\usepackage[decimalsymbol=comma]{siunitx}
%\usepackage{units}
%\usepackage[numbers]{natbib}
%\usepackage{pdfpages}
\usepackage{longtable}
%\usepackage{braket}
%\usepackage{color}
\usepackage[backend=biber]{biblatex}
\addbibresource{literature.bib}

%\renewcommand{\arraystretch}{1.2} % Standard-Zeilenhöhe einstellen

%%%%%%%%%%%%%%%%%%For tikz%%%%%%%%%%%%%%%%%%%%%%%%%%%%%%%%%
\usepackage{lmodern}
\usepackage{textcomp}
\usepackage{lastpage}
\usepackage{physics}
\usepackage{lipsum}
\usepackage{pxfonts}
\usepackage{pgfplots}
\usepackage{pgfplotstable}
\pgfplotsset{width=\textwidth, compat=1.17}
\usepgfplotslibrary{external} % LaTeX and plain TeX
\usetikzlibrary{pgfplots.external}
\usepgfplotslibrary[external] % LaTeX and plain TeX
\usetikzlibrary[pgfplots.external]
%%%%%%%%%%%%%%%%%%%%%%%%%%%%%%%%%%%%%%%%%%%%%%%%%%%%%%%%%%%

%---tables and imgaes
\usepackage{graphicx} %provide \includegraphics[options]{name}
%\usepackage{epstopdf} %enable the use of eps-graphics
\usepackage[hypcap]{caption} %captions out of floating-enviorments(figure,table)
\usepackage{booktabs} %extra lines in tabulars
%\usepackage{flafter} %Place floating-enviorments after references.
%\usepackage[ %
%section %latest ancor for floating-enviorments
%]{placeins}
\usepackage{verbatim} %long comments \begin{comment} ...\end{comment}

%---Hyperlinks
\usepackage{hyperref} %create table of content and references as links
\hypersetup{
%
%Colors
colorlinks=true,
breaklinks=true,
citecolor=blue,
linkcolor=black,
menucolor=red,
urlcolor=cyan,
%
%pdf-bookmarks
bookmarksopen=false,
bookmarksopenlevel=0,
%
%pdf-data
% pdftitle={\titel},
% pdfauthor={\writer},
% pdfcreator={\writer},
% pdfsubject={\titel},
% pdfkeywords={\titel}
%
%misc
plainpages=false,% zur korrekten Erstellung der Bookmarks
hypertexnames=false,% zur korrekten Erstellung der Bookmarks
% hyperindex=true,
}

